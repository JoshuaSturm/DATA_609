\documentclass[]{article}
\usepackage{lmodern}
\usepackage{amssymb,amsmath}
\usepackage{ifxetex,ifluatex}
\usepackage{fixltx2e} % provides \textsubscript
\ifnum 0\ifxetex 1\fi\ifluatex 1\fi=0 % if pdftex
  \usepackage[T1]{fontenc}
  \usepackage[utf8]{inputenc}
\else % if luatex or xelatex
  \ifxetex
    \usepackage{mathspec}
  \else
    \usepackage{fontspec}
  \fi
  \defaultfontfeatures{Ligatures=TeX,Scale=MatchLowercase}
\fi
% use upquote if available, for straight quotes in verbatim environments
\IfFileExists{upquote.sty}{\usepackage{upquote}}{}
% use microtype if available
\IfFileExists{microtype.sty}{%
\usepackage{microtype}
\UseMicrotypeSet[protrusion]{basicmath} % disable protrusion for tt fonts
}{}
\usepackage[margin=1in]{geometry}
\usepackage{hyperref}
\hypersetup{unicode=true,
            pdftitle={DATA 609 - Homework 6},
            pdfauthor={Joshua Sturm},
            pdfborder={0 0 0},
            breaklinks=true}
\urlstyle{same}  % don't use monospace font for urls
\usepackage{longtable,booktabs}
\usepackage{graphicx,grffile}
\makeatletter
\def\maxwidth{\ifdim\Gin@nat@width>\linewidth\linewidth\else\Gin@nat@width\fi}
\def\maxheight{\ifdim\Gin@nat@height>\textheight\textheight\else\Gin@nat@height\fi}
\makeatother
% Scale images if necessary, so that they will not overflow the page
% margins by default, and it is still possible to overwrite the defaults
% using explicit options in \includegraphics[width, height, ...]{}
\setkeys{Gin}{width=\maxwidth,height=\maxheight,keepaspectratio}
\IfFileExists{parskip.sty}{%
\usepackage{parskip}
}{% else
\setlength{\parindent}{0pt}
\setlength{\parskip}{6pt plus 2pt minus 1pt}
}
\setlength{\emergencystretch}{3em}  % prevent overfull lines
\providecommand{\tightlist}{%
  \setlength{\itemsep}{0pt}\setlength{\parskip}{0pt}}
\setcounter{secnumdepth}{0}
% Redefines (sub)paragraphs to behave more like sections
\ifx\paragraph\undefined\else
\let\oldparagraph\paragraph
\renewcommand{\paragraph}[1]{\oldparagraph{#1}\mbox{}}
\fi
\ifx\subparagraph\undefined\else
\let\oldsubparagraph\subparagraph
\renewcommand{\subparagraph}[1]{\oldsubparagraph{#1}\mbox{}}
\fi

%%% Use protect on footnotes to avoid problems with footnotes in titles
\let\rmarkdownfootnote\footnote%
\def\footnote{\protect\rmarkdownfootnote}

%%% Change title format to be more compact
\usepackage{titling}

% Create subtitle command for use in maketitle
\newcommand{\subtitle}[1]{
  \posttitle{
    \begin{center}\large#1\end{center}
    }
}

\setlength{\droptitle}{-2em}

  \title{DATA 609 - Homework 6}
    \pretitle{\vspace{\droptitle}\centering\huge}
  \posttitle{\par}
    \author{Joshua Sturm}
    \preauthor{\centering\large\emph}
  \postauthor{\par}
      \predate{\centering\large\emph}
  \postdate{\par}
    \date{October 25, 2018}


\begin{document}
\maketitle

\hypertarget{chapter-7-problems}{%
\section{Chapter 7 Problems}\label{chapter-7-problems}}

\hypertarget{page-251-exercise-2}{%
\subsection{1 (Page 251, exercise \#2)}\label{page-251-exercise-2}}

Nutritional Requirements-A rancher has determined that the minimum
weekly nutritional requirements for an average-sized horse include 40 lb
of protein, 20 lb of carbohydrates, and 45 lb of roughage. These are
obtained from the following sources in varying amounts at the prices
indicated:

\begin{longtable}[]{@{}lrrrl@{}}
\toprule
& Protein & Carbohydrates & Roughage & Cost\tabularnewline
\midrule
\endhead
Hay & 0.5 & 2.0 & 5.0 & 1.8\tabularnewline
Oats & 1.0 & 4.0 & 2.0 & 3.5\tabularnewline
Feeding blocks & 2.0 & 0.5 & 1.0 & 0.4\tabularnewline
High-protein concentrate & 6.0 & 1.0 & 2.5 & 1\tabularnewline
Requirements per horse & 40.0 & 20.0 & 45.0 &\tabularnewline
\bottomrule
\end{longtable}

Formulate a mathematical model to determine how to meet the minimum
nutritional requirements at minimum cost.

\hypertarget{solution}{%
\subsubsection{1 Solution}\label{solution}}

We are given four variables: \begin{align*}
x_1 &= \text{Hay} \\
x_2 &= \text{Oats} \\
x_3 &= \text{Feeding blocks} \\
x_4 &= \text{High-Protein concentrate}
\end{align*}

Our cost function is: \[
1.80x_1 + 3.5x_2 + 0.4x_3 + 1.0x_4
\]

And we want to minimize it, subject to the constraints: \begin{align*}
0.5x_1 + x_2 + 2.0x_3 + 6.0x_4 &\leq 40.0 \ \ \text(Protein) \\
2.0x_1 + 4.0x_2 + 0.5x_3 + 1.0x_4 &\leq 20.0 \ \ \text(Carbohydrates) \\
5.0x_1 + 2.0x_2 + 1.0x_3 + 2.5x_4 &\leq 45.0 \ \ \text(Roughage)
\end{align*}

\hypertarget{page-284-exercise-1}{%
\subsection{2 (Page 284, exercise \#1)}\label{page-284-exercise-1}}

For the example problem in this section, determine the sensitivity of
the optimal solution to a change in \(c_2\) using the objective function
\(25x_1 + c_2x_2\).

\hypertarget{solution-1}{%
\subsubsection{2 Solution}\label{solution-1}}

Let \(Z = 25x_1 + c_2x_2\).

Then \(x_1 = \frac{-c_2}{25}\).

The slope of the lumber constraint is \(\frac{-2}{3}\), and the slope of
the labour constraint is \(\frac{-5}{4}\).

Thus, the range of values for which the current extreme point remains
optimal is given by the inequality

\[
\frac{-5}{4} \leq \frac{-c_2}{25} \leq \frac{-2}{3}
\]

\[
\frac{50}{3} \leq c_2 \leq \frac{125}{4}
\]

\hypertarget{resources-used}{%
\section{Resources used:}\label{resources-used}}

\begin{itemize}
\tightlist
\item
  \url{http://sites.fas.harvard.edu/~apm121/lectures/lec8-hq.pdf}
\end{itemize}


\end{document}
