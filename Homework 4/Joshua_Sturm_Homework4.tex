\PassOptionsToPackage{unicode=true}{hyperref} % options for packages loaded elsewhere
\PassOptionsToPackage{hyphens}{url}
%
\documentclass[]{article}
\usepackage{lmodern}
\usepackage{amssymb,amsmath}
\usepackage{ifxetex,ifluatex}
\usepackage{fixltx2e} % provides \textsubscript
\ifnum 0\ifxetex 1\fi\ifluatex 1\fi=0 % if pdftex
  \usepackage[T1]{fontenc}
  \usepackage[utf8]{inputenc}
  \usepackage{textcomp} % provides euro and other symbols
\else % if luatex or xelatex
  \usepackage{unicode-math}
  \defaultfontfeatures{Ligatures=TeX,Scale=MatchLowercase}
\fi
% use upquote if available, for straight quotes in verbatim environments
\IfFileExists{upquote.sty}{\usepackage{upquote}}{}
% use microtype if available
\IfFileExists{microtype.sty}{%
\usepackage[]{microtype}
\UseMicrotypeSet[protrusion]{basicmath} % disable protrusion for tt fonts
}{}
\IfFileExists{parskip.sty}{%
\usepackage{parskip}
}{% else
\setlength{\parindent}{0pt}
\setlength{\parskip}{6pt plus 2pt minus 1pt}
}
\usepackage{hyperref}
\hypersetup{
            pdftitle={DATA 609 - Homework 4},
            pdfauthor={Joshua Sturm},
            pdfborder={0 0 0},
            breaklinks=true}
\urlstyle{same}  % don't use monospace font for urls
\setlength{\emergencystretch}{3em}  % prevent overfull lines
\providecommand{\tightlist}{%
  \setlength{\itemsep}{0pt}\setlength{\parskip}{0pt}}
\setcounter{secnumdepth}{0}
% Redefines (sub)paragraphs to behave more like sections
\ifx\paragraph\undefined\else
\let\oldparagraph\paragraph
\renewcommand{\paragraph}[1]{\oldparagraph{#1}\mbox{}}
\fi
\ifx\subparagraph\undefined\else
\let\oldsubparagraph\subparagraph
\renewcommand{\subparagraph}[1]{\oldsubparagraph{#1}\mbox{}}
\fi

% set default figure placement to htbp
\makeatletter
\def\fps@figure{htbp}
\makeatother

\usepackage{float}
\usepackage[table]{xcolor}
\usepackage{booktabs}
\usepackage{makecell}

\title{DATA 609 - Homework 4}
\author{Joshua Sturm}
\date{October 3, 2018}

\begin{document}
\maketitle

\hypertarget{chapter-5-problems}{%
\section{Chapter 5 problems}\label{chapter-5-problems}}

\hypertarget{page-191-exercise-3}{%
\subsection{1 (Page 191, exercise \#3)}\label{page-191-exercise-3}}

Using Monte Carlo simulation, write an algorithm to calculate an
approximation to \(\pi\) by considering the number of random points
selected inside the quarter circle

\begin{equation*}
Q : x^2 + y^2 = 1, \ \ x \geq 0, \ y \geq 0
\end{equation*}

where the quarter circle is taken to be inside the square

\begin{equation*}
S : 0 \leq x \leq 1 \ \text{and} \ 0 \leq y \leq 1
\end{equation*}

Use the equation \(\frac{\pi}{4} = \frac{\text{area}Q}{\text{area}S}\)

\hypertarget{solution}{%
\subsubsection{1 Solution}\label{solution}}

\begin{table}[H]
\centering\rowcolors{2}{gray!6}{white}

\begin{tabular}{rrrr}
\hiderowcolors
\toprule
\textbf{Sample Size} & \textbf{Actual $\pi$} & \textbf{Estimated $\pi$} & \textbf{Absolute difference}\\
\midrule
\showrowcolors
10 & 3.141593 & 2.80000 & 0.3415927\\
50 & 3.141593 & 3.12000 & 0.0215927\\
100 & 3.141593 & 3.20000 & 0.0584073\\
500 & 3.141593 & 3.19200 & 0.0504073\\
1000 & 3.141593 & 3.12800 & 0.0135927\\
\addlinespace
5000 & 3.141593 & 3.11040 & 0.0311927\\
10000 & 3.141593 & 3.13280 & 0.0087927\\
50000 & 3.141593 & 3.13520 & 0.0063927\\
100000 & 3.141593 & 3.14772 & 0.0061273\\
\bottomrule
\end{tabular}
\rowcolors{2}{white}{white}
\end{table}

As the sample size increases, the simulation's accuracy improves.

\hypertarget{page-194-exercise-1}{%
\subsection{2 (Page 194, exercise \#1)}\label{page-194-exercise-1}}

Use the middle-square method to generate \begin{align*}
&a. \text{ 10 random numbers using } x_0 = 1009.\\
&b. \text{ 20 random numbers using } x_0 = 653217.\\
&c. \text{ 15 random numbers using } x_0 = 3043.\\
&d. \text{ Comment about the results of each sequence. Was there cycling? Did each sequence degenerate rapidly?}
\end{align*}

\hypertarget{solution-1}{%
\subsubsection{2 Solution}\label{solution-1}}

\end{document}
